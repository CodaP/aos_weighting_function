%%%%%%%%%%%%%%%%%%%%%%%%%%%%%%%%%%%%%%%%%%%%%%%%%%%%%%%%%%%%%%%%%%%%%%
% amspaper.tex --  LaTeX-based template for submissions to American 
% Meteorological Society journals
%
% Template developed by Amy Hendrickson, 2013, TeXnology Inc., 
% amyh@texnology.com, http://www.texnology.com
% following earlier work by Brian Papa, American Meteorological Society
%
% Email questions to latex@ametsoc.org.
%
%%%%%%%%%%%%%%%%%%%%%%%%%%%%%%%%%%%%%%%%%%%%%%%%%%%%%%%%%%%%%%%%%%%%%
% PREAMBLE
%%%%%%%%%%%%%%%%%%%%%%%%%%%%%%%%%%%%%%%%%%%%%%%%%%%%%%%%%%%%%%%%%%%%%

%% Start with one of the following:
% DOUBLE-SPACED VERSION FOR SUBMISSION TO THE AMS
% \documentclass{ametsoc}

% TWO-COLUMN JOURNAL PAGE LAYOUT---FOR AUTHOR USE ONLY
\documentclass[twocol]{ametsoc}

%%%%%%%%%%%%%%%%%%%%%%%%%%%%%%%%
%%% To be entered only if twocol option is used

\journal{jamc}
\usepackage{gensymb}
\usepackage{graphicx}
\usepackage{booktabs}
\usepackage{hyperref}

\DeclareGraphicsExtensions{.pdf,.png,.jpg}
%  Please choose a journal abbreviation to use above from the following list:
% 
%   jamc     (Journal of Applied Meteorology and Climatology)
%   jtech     (Journal of Atmospheric and Oceanic Technology)
%   jhm      (Journal of Hydrometeorology)
%   jpo     (Journal of Physical Oceanography)
%   jas      (Journal of Atmospheric Sciences)	
%   jcli      (Journal of Climate)
%   mwr      (Monthly Weather Review)
%   wcas      (Weather, Climate, and Society)
%   waf       (Weather and Forecasting)
%   bams (Bulletin of the American Meteorological Society)
%   ei    (Earth Interactions)

%%%%%%%%%%%%%%%%%%%%%%%%%%%%%%%%
%Citations should be of the form ``author year''  not ``author, year''
\bibpunct{(}{)}{;}{a}{}{,}

%%%%%%%%%%%%%%%%%%%%%%%%%%%%%%%%

%%% To be entered by author:

%% May use \\ to break lines in title:

\title{Weighting Function Project}

%%% Enter authors' names, as you see in this example:
%%% Use \correspondingauthor{} and \thanks{Current Affiliation:...}
%%% immediately following the appropriate author.
%%%
%%% Note that the \correspondingauthor{} command is NECESSARY.
%%% The \thanks{} commands are OPTIONAL.

    \authors{Coda Phillips\correspondingauthor{Coda Phillips, 
     University of Wisconsin - Madison 
     }}

     \affiliation{University of Wisconsin - Madison}

\email{codaphillips@gmail.com}



%%%%%%%%%%%%%%%%%%%%%%%%%%%%%%%%%%%%%%%%%%%%%%%%%%%%%%%%%%%%%%%%%%%%%
% ABSTRACT
%
% Enter your Abstract here

\abstract{I make a radiative transfer model for AMSU and study things} 

\begin{document}


%% Necessary!
\maketitle


%%%%%%%%%%%%%%%%%%%%%%%%%%%%%%%%%%%%%%%%%%%%%%%%%%%%%%%%%%%%%%%%%%%%%
% MAIN BODY OF PAPER
%%%%%%%%%%%%%%%%%%%%%%%%%%%%%%%%%%%%%%%%%%%%%%%%%%%%%%%%%%%%%%%%%%%%%
%
\section{Introduction}

Atmospheric radiation is an important component of global climate.
The ability to study the effects of changes in distribution and balance of radiation are necessary for developing accurate predictive models and forecasts.
Global climate models rely on the efficient computation of radiative transfer at hundreds of points around the globe and many atmospheric levels at each point.
Passive instruments such as AMSU and CrIS require a radiative transfer model to invert radiation measurements and derive atmospheric environmental data such as temperature profiles and water vapor content.
\par In this paper, a simple radiative transfer model is constructed for the purpose of studying the effects of environmental permutations on AMSU observation.
For input, this RTM receives a  single radiosonde file containing pressure, temperature, and dew point information.
By superposition of the radiative absorption from dry air and water vapor, an optical depth can be computed at each layer of atmosphere.
These optical depths are integrated and transformed into total monochromatic transmission from either the top-of-atmosphere or the surface.
By combining these derived opacities with atmospheric and surface temperature information, the radiation observed from a space-based instrument like AMSU can be predicted and compared to real observations.

\section{Methods}

For all experiments, surface temperature was assumed equal to the temperature of the lowest radiosonde measurement.
In all cases except the emissivity experiments, the surface was also assumed to radiate as a blackbody at thermal equilibrium. That is, the emissivity is equal to 1 at all wavelengths.
The frequencies corresponding to the referenced AMSU channels are defined in \autoref{tab:amsua} and \autoref{tab:amsub}.

\begin{table}
	\centering
	\caption{AMSU-A Channels}
	\begin{tabular}{cc}
		\toprule
		Channel	& Frequency (GHz)\\
		\midrule
	    3 & 50.3000 \\
	    4 & 52.8000 \\
	    5 & 53.7110 \\
	    6 & 54.4000 \\
	    7 & 54.9400 \\
	    8 & 55.5000 \\
	    9 & 57.2900 \\
	    10 & 57.5070 \\
	    \bottomrule
	\end{tabular}
	\label{tab:amsua}
\end{table}

\begin{table}
	\centering
	\caption{AMSU-B Channels}
	\begin{tabular}{cc}
		\toprule
		Channel	& Frequency (GHz)\\
		\midrule
		16 & 89.00 \\
		17 & 150.0 \\
		18 & 184.31 \\
		19 & 186.31 \\
		20 & 190.31 \\
	    \bottomrule
	\end{tabular}
	\label{tab:amsub}
\end{table}

\subsection{Atmospheric Profile}

Before an atmospheric profile can be computed, environmental data must be ingested from a radiosonde file.
Contained in the radiosonde data is a time-series of temperatures, dew points, and pressures recorded as the radiosonde ascends. The state of the atmosphere is assumed to be static so that a these measurements describe a vertical profile. This profile is interpolated to a finer resolution and a more uniform grid by ensuring each successive level falls within a maximum pressure difference of 90\% and maximum temperature change of 1 \degree{}C.

\subsection{Layers}

A model atmosphere is partitioned into homogeneous atmospheric layers, each possessing the mean value of two level-variables and a geometric height difference between levels.
The geometric height associated with each layer is a required to calculate optical depths, and these are derived from a form of the hydrostatic equation rather than being obtained from the radiosonde's GPS altitude.

\subsection{Absorption}

The total optical depth, $\tau$, of each layer is computed by adding optical depth components from dry air and water vapor.
These components are computed individually from the product of the partial mass of each gas and its mass extinction coefficient.
The mass extinction coefficient is dependent on temperature, pressure, and water vapor mixing ratio.

\subsection{Transmission}

An integral over optical depths is performed by computing a cumulative sum.
According to Beer's law, the antilogarithm of these depths results in the total transmission to that layer from either the top-of-atmosphere or surface, depending on the direction of the cumulative sum.
By Schwarzschild's equation, the weighting function takes the value of the derivative of transmission at each level.
A discrete difference is calculated to approximate this derivative.

\subsection{Weights}

The resulting weights associated with each layer can finally be divided by geometric depth their respective layers to derive the weight as a function of geometric depth.

\subsection{Experiments}

\paragraph*{Case Study} For the following experiments, two cases of atmosphere will be compared.
One case will study the Arctic atmosphere and the other will study the Tropical.
The Arctic case uses a radiosonde launched from Churchill, Canada at 12Z on the first of March, 2012.
The Tropical radiosonde was launched from Boa Vista, Brazil, at the same date and time.

\paragraph{Low frequency channels (3-10) Module A }

Another characteristic of the weighting functions is the relative strengths of the weights.
The area under each weighting function is equal to the total absorption from the surface to the top of the atmosphere.
 
\paragraph{High frequency channels (18-20)}

For a baseline test, the weighting functions for each of the channels in the AMSU-A module.
These are computed for both locations and the results are shown in \autoref{fig:weighting_a}.

\paragraph{Spectra}

If rather than computing weights just for AMSU channels, they are computed for the entire spectrum, the observed brightness temperatures can be predicted.
This is done by taking the dot product of the layer weights and the temperature profile and adding the transmission multiplied by surface brightness temperature.

\paragraph{Emissivity}
For both the Brazilian and Canadian soundings, surface emissivity is halved and the resulting spectra was plotted.

\paragraph{Elevation angle}
For both the Brazilian and Canadian soundings, the path through the atmosphere was modified from the default zenith angle to a slant of 55\degree{} from vertical.
The resulting spectra was plotted.

\section{Results}

\subsection{AMSU-A Weighting Functions}

There are two main differences seen in the weighting functions produced by a Tropical and Arctic atmosphere.
The first difference is the location of the most visible layer.
Because it has the greatest effect on the transmission, the most visible layer appears on the weighting function as a peak.
Variations in the temperature of this most visible layer has the most effect on radiation escaping to space.
Therefore, the altitude of this layer plays a role in determining the output radiation.
Here, the Canadian weighting functions seem to have a depressed peak altitude when compared with the Brazilian case.
Depending on the temperature lapse rate, the relative locations of the most visible layers can either increase or decrease the radiation output at the top-of-atmosphere.

Shown in \autoref{fig:weighting_a} are the weighting functions for the AMSU-A channels for both the Canadian and Brazilian cases.
These results show some similarity between the cases in the relative magnitudes and distributions of weights.
However, the peak weights in Canadian case almost always occur at a lower altitude than those of the Brazilian case.
These depressed heights indicate a comparatively more transparent Canadian atmosphere.
Some channels are much more opaque than others, despite all inhabiting a relatively narrow bandwidth.

Total absorption dictates the influence that the surface has in the radiative output.
As shown in the graph, Canada has less area under each curve than Brazil, and therefore a higher transmission.

\begin{figure}
	\centering
	\includegraphics[width=.5\textwidth]{figures/weighting_a}
	\caption{thing1}
	\label{fig:weighting_a}
\end{figure}

\subsection{AMSU-B Weighting Functions}

Compared to the low-frequency AMSU-A channels, the AMSU-B channels resolve much more profound differences between the two locations.
The weighting functions are shown in \autoref{fig:weighting_b}.
The Canadian weights are clustered at heights below 5~km, whereas Brazilian weights tend to appear above 5~km.
It is also apparent that Canadian weights cover much less area than Brazil, indicating that transmission is dramatically improved in the Canadian case.
The weights shown in \autoref{fig:weighting_b} demonstrate that little absorption occurs in the upper atmosphere at these higher frequencies.
The difference between the weight-peak altitude in the Brazilian and Canadian atmospheres is much more pronounced here than with the AMSU-A channels.
These data would suggest that although a high amount of absorption occurs, it is relegated to the troposphere.
It is likely that almost all absorption in these bands is driven by water vapor, which explains why radiation is able to propagate farther in the dry Canadian atmosphere before it is extinguished.

\begin{figure}
	\centering
	\includegraphics[width=.5\textwidth]{figures/weighting_b}
	\caption{thing2}
	\label{fig:weighting_b}
\end{figure}

\subsection{Spectra}

The predicted spectra for Brazil and Canada is shown in \autoref{fig:spec}.
Excluding the strongest absorption lines, the predicted brightness temperatures observed from space for the Brazilian and Canadian cases differ significantly.
The average values of brightness temperature for the Brazilian case are nearly 40~K greater than those from Canada.
Additionally, there is a downward slope of brightness temperatures from Brazil as frequency increases.
This effect is not apparent in the Canadian case.
The lack of continuum absorption in the Canadian atmosphere is supported by the fact that little water vapor is present in that case.
Furthermore, the absorption band of oxygen at 22~GHz does not produce a noticeable effect in the Canadian brightness temperatures, but it appears as a small dip in those from Brazil.

Oxygen is responsible for the bands seen at 20, 60, and 118 GHz, and water vapor absorbs strongly at 183~GHz. Water vapor also weakly absorbs on a continuum with higher frequencies being absorbed more.
Within the strongest absorption lines, these spectra look remarkably similar.

\begin{figure}
	\centering
	\includegraphics[width=.5\textwidth]{figures/spec}
	\caption{spectrum}
	\label{fig:spec}
\end{figure}

\subsubsection{Surface Temperature}

The Brazilian surface temperature is 296.75~\degree{K}, whereas the Canadian surface temperature is 254.05~\degree{K}.
The surface temperature in Canada is much colder than in Brazil.
This effect is evident in the satellite spectrum at the lowest frequencies, where the atmosphere should be transparent.
The resulting brightness temperatures are nearly identical to the actual temperatures.
This correlation would suggest that both atmospheres are highly transparent in these bands, and changes in brightness temperature emitted from the surface produce a strong effect in predicted brightness temperatures.
However, surface temperature cannot explain the difference in brightness temperature response to frequency or the values within absorption bands.

\subsubsection{Temperature Profile}

It appears that Canadian temperatures are warmer than Brazilian temperatures at 17.5~km.
But, as expected, temperatures are colder in Canada below tropopause.
In the weighting functions you can see the altitude contributing the most to absorption and emission. Therefore, brightness temperatures will tend toward the physical temperature of that particular layer rather than the others.
In channel~18, corresponding to 184~GHz, the altitude of peak absorption is located at about 10~km in Brazil and 3~km in Canada.
If brightness temperature is read at 184~GHz, it comes out to about 240~K for both Brazil and Canada.
This matches the temperatures in the profile measured at the altitudes of peak emission, which confirms that the radiative transfer model is consistent.

The differences in brightness temperature spectra within the strong absorption bands can be partially explained by the differences in temperature profile.
Most Canadian brightness temperatures observed within these bands appear to be greater than those from Brazil.
By examining \autoref{fig:weighting_a}, we see that the weighting functions, for the higher frequency AMSU-A channels pertaining to the first strong absorption band, are aligned between Brazil and Canada.
Therefore, differences in observed brightness temperatures in these channels should only be caused by differences in emission at the most highly weighted layers.
The evidence of warmer Canadian brightness temperatures is consistent with warmer Canadian temperatures in the stratosphere and the stratospheric layers receiving the highest weight at that channel.

\begin{figure}
	\centering
	\includegraphics[width=.5\textwidth]{figures/tdry}
	\caption{Temperature}
	\label{fig:tdry}
\end{figure}

\subsubsection{Humidity}

\autoref{fig:qbar} shows that the Brazilian radiosonde measured much more moisture than the Canadian.
Considering the entire atmosphere, the Brazilian atmosphere was found to contain 55.22~kg/$m^2$ water vapor, and the Canadian atmosphere only contained 3.12~kg/$m^2$.
The profile confirms that water vapor is concentrated in the troposphere for both locations and peaks at lowest altitudes.
Water vapor is a strong absorber, it has one particularly strong band at 183 GHz.
As shown in \autoref{fig:spec}, a dip in brightness temperature occurs at this frequency in both locations.
By referencing \autoref{tab:amsub} and \autoref{fig:weighting_b} we see that the brightness temperature measured from space is primarily influenced by the air temperature at a height of 10~km in Brazil and 4~km in Canada.
According to \autoref{fig:tdry}, these altitudes correspond to about 240~K for Brazil and 250~K for Canada.
The brightness temperature at the local minima of the 183~GHz water vapor dip agrees with these values.


You can see the Brazilian brightness temperatures diminish as continuum absorption increases with higher frequencies.
Outside major oxygen and water vapor absorption bands, absorption is mostly governed by the water vapor continuum absorption.
Absorption and emission by water vapor usually occurs at cooler temperatures than surface emission.
Therefore, partial absorption creates a cold bias in the predicted brightness temperatures, and this bias strengthens as absorption strengthens.
The baseline brightness temperatures in the Brazilian case remains relatively constant as a result of having little water vapor.

\begin{figure}
	\centering
	\includegraphics[width=.5\textwidth]{figures/qbar}
	\caption{Specific Humidity}
	\label{fig:qbar}
\end{figure}

\subsection{Emissivity}


\begin{figure}
	\centering
	\includegraphics[width=.5\textwidth]{figures/brazil_emiss}
	\caption{Brazil emissivity}
	\label{fig:bemiss}
\end{figure}


\begin{figure}
	\centering
	\includegraphics[width=.5\textwidth]{figures/canada_emiss}
	\caption{Canada emissivity}
	\label{fig:cemiss}
\end{figure}

\subsubsection{Brazil}

The emissivity experiment results for Brazil in \autoref{fig:bemiss} show significant differences at the lower frequencies after changing the surface emissivity.
At frequencies above 150~GHz, the surface emissivity has little effect on predicted brightness temperatures observed from space.
There is also little effect within the strong absorption bands.
This makes sense because higher absorption at high frequencies and absorption bands obscures the surface and masks the effect of emissivity.
In bands of high transmission, halving the emissivity approximately halves brightness temperature.
Therefore, brightness temperatures at the lowest frequencies, where the atmosphere is nearly transparent, appear to be halved.
In between the two extremes are regions of partial absorption, which create a warm bias in the observed brightness temperatures because the model atmosphere is non-scattering and radiates as a blackbody.

\subsubsection{Canada}

The results of halving emissivity in the Canadian case are shown in \autoref{fig:cemiss}.
Here we see that the atmosphere is mostly transparent in all bands except the specific strongly-absorbing bands.
The continuum absorbing effect of water vapor is now evident.
This is because the atmosphere and surface are now radiating at substantially different brightness temperatures and additional absorption and emission by the atmosphere is contrasted by a very cold surface. 

\subsection{Elevation}

The effect of changing elevation angle in the Brazilian case are shown in \autoref{fig:elevation}.
We have seen that one of the most important components of the predicted spectra is surface emission.
Although increasing the slant angle increases the optical depth of each absorbing layer and equates to an inflation of the weighting-peak altitudes, the effects shown are most likely caused by a decrease in transmission.
The increase in weighting-peak height is shown in \autoref{fig:elevation_weight}, and usually amounts to about 3~km.
In the lowest highest frequencies in the AMSU-A bands this actually increases brightness temperatures.
However, the resulting temperature would still be much less than the observed brightness temperature.
Therefore, the depression in brightness temperatures observed in \autoref{fig:elevation} is probably due to reduced transmission from the surface.
This is still captured in \autoref{fig:elevation_weight} by the comparatively larger area under each curve at 55\degree{} elevation when compared to nadir.

\begin{figure}
	\centering
	\includegraphics[width=.5\textwidth]{figures/brazil_elevation}
	\caption{Elevation angle}
	\label{fig:elevation}
\end{figure}

\begin{figure}
	\centering
	\includegraphics[width=.5\textwidth]{figures/elevation_weights}
	\caption{Elevation weighting function}
	\label{fig:elevation_weight}
\end{figure}

\section{Conclusion}

From the results it was shown that the brightness temperature received in space is primarily a composition of two sources.
First there is the surface emission
Depends on the emissivity of surface and the opacity of the atmosphere.
Then there is the atmospheric emission
Depends on the temperature and opacity of atmosphere layers.

%%%%%%%%%%%%%%%%%%%%%%%%%%%%%%%%%%%%%%%%%%%%%%%%%%%%%%%%%%%%%%%%%%%%%
% REFERENCES
%%%%%%%%%%%%%%%%%%%%%%%%%%%%%%%%%%%%%%%%%%%%%%%%%%%%%%%%%%%%%%%%%%%%%

 \bibliographystyle{ametsoc2014}
 \bibliography{references}

\end{document}
%%%%%%%%%%%%%%%%%%%%%%%%%%%%%%%%%%%%%%%%%%%%%%%%%%%%%%%%%%%%%%%%%%%%%
% END OF AMSPAPER.TEX
%%%%%%%%%%%%%%%%%%%%%%%%%%%%%%%%%%%%%%%%%%%%%%%%%%%%%%%%%%%%%%%%%%%%%