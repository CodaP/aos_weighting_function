%%%%%%%%%%%%%%%%%%%%%%%%%%%%%%%%%%%%%%%%%%%%%%%%%%%%%%%%%%%%%%%%%%%%%%
% amspaper.tex --  LaTeX-based template for submissions to American 
% Meteorological Society journals
%
% Template developed by Amy Hendrickson, 2013, TeXnology Inc., 
% amyh@texnology.com, http://www.texnology.com
% following earlier work by Brian Papa, American Meteorological Society
%
% Email questions to latex@ametsoc.org.
%
%%%%%%%%%%%%%%%%%%%%%%%%%%%%%%%%%%%%%%%%%%%%%%%%%%%%%%%%%%%%%%%%%%%%%
% PREAMBLE
%%%%%%%%%%%%%%%%%%%%%%%%%%%%%%%%%%%%%%%%%%%%%%%%%%%%%%%%%%%%%%%%%%%%%

%% Start with one of the following:
% DOUBLE-SPACED VERSION FOR SUBMISSION TO THE AMS
% \documentclass{ametsoc}

% TWO-COLUMN JOURNAL PAGE LAYOUT---FOR AUTHOR USE ONLY
\documentclass[twocol]{ametsoc}

%%%%%%%%%%%%%%%%%%%%%%%%%%%%%%%%
%%% To be entered only if twocol option is used

\journal{jamc}
\usepackage{gensymb}
\usepackage{graphicx} 

\DeclareGraphicsExtensions{.pdf,.png,.jpg}
%  Please choose a journal abbreviation to use above from the following list:
% 
%   jamc     (Journal of Applied Meteorology and Climatology)
%   jtech     (Journal of Atmospheric and Oceanic Technology)
%   jhm      (Journal of Hydrometeorology)
%   jpo     (Journal of Physical Oceanography)
%   jas      (Journal of Atmospheric Sciences)	
%   jcli      (Journal of Climate)
%   mwr      (Monthly Weather Review)
%   wcas      (Weather, Climate, and Society)
%   waf       (Weather and Forecasting)
%   bams (Bulletin of the American Meteorological Society)
%   ei    (Earth Interactions)

%%%%%%%%%%%%%%%%%%%%%%%%%%%%%%%%
%Citations should be of the form ``author year''  not ``author, year''
\bibpunct{(}{)}{;}{a}{}{,}

%%%%%%%%%%%%%%%%%%%%%%%%%%%%%%%%

%%% To be entered by author:

%% May use \\ to break lines in title:

\title{Weighting Function Project}

%%% Enter authors' names, as you see in this example:
%%% Use \correspondingauthor{} and \thanks{Current Affiliation:...}
%%% immediately following the appropriate author.
%%%
%%% Note that the \correspondingauthor{} command is NECESSARY.
%%% The \thanks{} commands are OPTIONAL.

    \authors{Coda Phillips\correspondingauthor{Coda Phillips, 
     University of Wisconsin - Madison 
     }}

     \affiliation{University of Wisconsin - Madison}

\email{codaphillips@gmail.com}



%%%%%%%%%%%%%%%%%%%%%%%%%%%%%%%%%%%%%%%%%%%%%%%%%%%%%%%%%%%%%%%%%%%%%
% ABSTRACT
%
% Enter your Abstract here

\abstract{I make a radiative transfer model for AMSU and study things} 

\begin{document}


%% Necessary!
\maketitle


%%%%%%%%%%%%%%%%%%%%%%%%%%%%%%%%%%%%%%%%%%%%%%%%%%%%%%%%%%%%%%%%%%%%%
% MAIN BODY OF PAPER
%%%%%%%%%%%%%%%%%%%%%%%%%%%%%%%%%%%%%%%%%%%%%%%%%%%%%%%%%%%%%%%%%%%%%
%
\section{Introduction}

Atmospheric radiation is an important component of global climate.
The ability to study the effects of changes in distribution and balance of radiation are to developing accurate predictive models and forecasts.
Global climate models rely on the efficient computation of radiative transfer at hundreds of points around the globe and many atmospheric levels at each point.
Passive instruments such as AMSU and CrIS require a radiative transfer model to invert radiation measurements and derive atmospheric environmental data such as temperature profiles and water vapor content.
\par In this paper, a simple radiative transfer model is constructed for the purpose of studying the effects of environmental permutations on AMSU observation.
For input, this RTM receives a  single radiosonde file containing pressure, temperature, and dew point information.
By superposition of the radiative absorption from dry air and water vapor, an optical depth can be computed at each layer of atmosphere.
These optical depths are integrated and transformed to total monochromatic transmission from either the top-of-atmosphere or the surface.

\section{Methods}

\subsection{Atmospheric Profile}

For all experiments, surface temperature was assumed equal to the mean temperature of the 

\subsection{Layers}

\subsection{Absorption}

\subsection{Transmission}

\subsection{Weights}

\subsection{Experiments}

\paragraph*{Case Study} Canada vs Brazil

\paragraph{Low frequency channels (3-10) Module A }
 Compute and plot the weighting functions for channels 3-10 for both soundings (same plot, use colors and/or dashes to distinguish between soundings).  How do the weighting functions differ between the two cases?
 
\paragraph{High frequency channels (18-20)}
Do the same for channels 18-20.

\paragraph{Spectra}
Compute and plot (on the same plot) the complete spectra of microwave brightness temperature for the two soundings — 0.1 to 300 GHz at intervals of 0.1 GHz.   Describe the differences in results. 
Attribute differences to (a) differences in the surface temperature, (b) differences in the atmospheric temperature profile, and (c) differences in the atmospheric humidity profile.

\paragraph{Emissivity}
For just the tropical sounding, plot the spectra for two cases:  emissivity equal to 1  and emissivity equal to 0.5.  What do your results tell you about the ranges of frequencies for which the atmosphere is opaque?
Repeat the above for the polar sounding and describe what changes relative to the previous results.

\paragraph{Elevation angle}
Again, plot spectra for the tropical sounding, this time for two nadir angles θ:  0 degrees and 55 degrees.  Discuss the resulting differences.

\section{Results}

\subsection{Weighting Functions A}
Compute and plot the weighting functions for channels 3-10 for both soundings (same plot, use colors and/or dashes to distinguish between soundings).  How do the weighting functions differ between the two cases?

\begin{figure}
	\centering
	\includegraphics[width=.5\textwidth]{figures/weighting_a}
	\caption{thing1}
	\label{fig:weighting_a}
\end{figure}

\subsection{Weighting Functions B}
Do the same for channels 18-20.

\begin{figure}
	\centering
	\includegraphics[width=.5\textwidth]{figures/weighting_b}
	\caption{thing2}
	\label{fig:weighting_b}
\end{figure}

\subsection{Spectra}
Compute and plot (on the same plot) the complete spectra of microwave brightness temperature for the two soundings — 0.1 to 300 GHz at intervals of 0.1 GHz.   Describe the differences in results.


\begin{figure}
	\centering
	\includegraphics[width=.5\textwidth]{figures/spec}
	\caption{spectrum}
	\label{fig:spec}
\end{figure}

\subsubsection{Surface Temperature}

\subsubsection{Temperature Profile}


\begin{figure}
	\centering
	\includegraphics[width=.5\textwidth]{figures/tdry}
	\caption{Temperature}
	\label{fig:tdry}
\end{figure}

(b) differences in the atmospheric temperature profile
\subsubsection{Humidity}
(c) differences in the atmospheric humidity profile.

\begin{figure}
	\centering
	\includegraphics[width=.5\textwidth]{figures/qbar}
	\caption{Specific Humidity}
	\label{fig:qbar}
\end{figure}

\subsection{Emissivity}


\begin{figure}
	\centering
	\includegraphics[width=.5\textwidth]{figures/brazil_emiss}
	\caption{Brazil emissivity}
	\label{fig:bemiss}
\end{figure}


\begin{figure}
	\centering
	\includegraphics[width=.5\textwidth]{figures/canada_emiss}
	\caption{Canada emissivity}
	\label{fig:cemiss}
\end{figure}

\subsubsection{$\epsilon = 1$}
For just the tropical sounding, plot the spectra for two cases:  emissivity equal to 1  and emissivity equal to 0.5.
\subsubsection{$\epsilon = 0.5$}
What do your results tell you about the ranges of frequencies for which the atmosphere is opaque?
Repeat the above for the polar sounding and describe what changes relative to the previous results.

\subsection{Elevation}


\begin{figure}
	\centering
	\includegraphics[width=.5\textwidth]{figures/brazil_elevation}
	\caption{Elevation angle}
	\label{fig:elevation}
\end{figure}

Again, plot spectra for the tropical sounding, this time for two nadir angles θ:  0 degrees and 55 degrees.  Discuss the resulting differences.
\subsubsection{$\theta = 0 \degree{} $}
\subsubsection{$\theta = 55 \degree{} $}


\section{Conclusion}

%%%%%%%%%%%%%%%%%%%%%%%%%%%%%%%%%%%%%%%%%%%%%%%%%%%%%%%%%%%%%%%%%%%%%
% REFERENCES
%%%%%%%%%%%%%%%%%%%%%%%%%%%%%%%%%%%%%%%%%%%%%%%%%%%%%%%%%%%%%%%%%%%%%
 This shows how to enter the commands for making a bibliography using
 BibTeX. It uses references.bib and the ametsoc2014.bst file for the style.

 \bibliographystyle{ametsoc2014}
 \bibliography{references}

\end{document}
%%%%%%%%%%%%%%%%%%%%%%%%%%%%%%%%%%%%%%%%%%%%%%%%%%%%%%%%%%%%%%%%%%%%%
% END OF AMSPAPER.TEX
%%%%%%%%%%%%%%%%%%%%%%%%%%%%%%%%%%%%%%%%%%%%%%%%%%%%%%%%%%%%%%%%%%%%%