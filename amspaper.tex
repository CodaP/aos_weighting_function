%%%%%%%%%%%%%%%%%%%%%%%%%%%%%%%%%%%%%%%%%%%%%%%%%%%%%%%%%%%%%%%%%%%%%%
% amspaper.tex --  LaTeX-based template for submissions to American 
% Meteorological Society journals
%
% Template developed by Amy Hendrickson, 2013, TeXnology Inc., 
% amyh@texnology.com, http://www.texnology.com
% following earlier work by Brian Papa, American Meteorological Society
%
% Email questions to latex@ametsoc.org.
%
%%%%%%%%%%%%%%%%%%%%%%%%%%%%%%%%%%%%%%%%%%%%%%%%%%%%%%%%%%%%%%%%%%%%%
% PREAMBLE
%%%%%%%%%%%%%%%%%%%%%%%%%%%%%%%%%%%%%%%%%%%%%%%%%%%%%%%%%%%%%%%%%%%%%

%% Start with one of the following:
% DOUBLE-SPACED VERSION FOR SUBMISSION TO THE AMS
% \documentclass{ametsoc}

% TWO-COLUMN JOURNAL PAGE LAYOUT---FOR AUTHOR USE ONLY
\documentclass[twocol]{ametsoc}

%%%%%%%%%%%%%%%%%%%%%%%%%%%%%%%%
%%% To be entered only if twocol option is used

\journal{jamc}

%  Please choose a journal abbreviation to use above from the following list:
% 
%   jamc     (Journal of Applied Meteorology and Climatology)
%   jtech     (Journal of Atmospheric and Oceanic Technology)
%   jhm      (Journal of Hydrometeorology)
%   jpo     (Journal of Physical Oceanography)
%   jas      (Journal of Atmospheric Sciences)	
%   jcli      (Journal of Climate)
%   mwr      (Monthly Weather Review)
%   wcas      (Weather, Climate, and Society)
%   waf       (Weather and Forecasting)
%   bams (Bulletin of the American Meteorological Society)
%   ei    (Earth Interactions)

%%%%%%%%%%%%%%%%%%%%%%%%%%%%%%%%
%Citations should be of the form ``author year''  not ``author, year''
\bibpunct{(}{)}{;}{a}{}{,}

%%%%%%%%%%%%%%%%%%%%%%%%%%%%%%%%

%%% To be entered by author:

%% May use \\ to break lines in title:

\title{Weighting Function Project}

%%% Enter authors' names, as you see in this example:
%%% Use \correspondingauthor{} and \thanks{Current Affiliation:...}
%%% immediately following the appropriate author.
%%%
%%% Note that the \correspondingauthor{} command is NECESSARY.
%%% The \thanks{} commands are OPTIONAL.

    \authors{Coda Phillips\correspondingauthor{Coda Phillips, 
     University of Wisconsin - Madison 
     }}

     \affiliation{University of Wisconsin - Madison}

\email{codaphillips@gmail.com}



%%%%%%%%%%%%%%%%%%%%%%%%%%%%%%%%%%%%%%%%%%%%%%%%%%%%%%%%%%%%%%%%%%%%%
% ABSTRACT
%
% Enter your Abstract here

\abstract{I make a radiative transfer model for AMSU and study things} 

\begin{document}


%% Necessary!
\maketitle


%%%%%%%%%%%%%%%%%%%%%%%%%%%%%%%%%%%%%%%%%%%%%%%%%%%%%%%%%%%%%%%%%%%%%
% MAIN BODY OF PAPER
%%%%%%%%%%%%%%%%%%%%%%%%%%%%%%%%%%%%%%%%%%%%%%%%%%%%%%%%%%%%%%%%%%%%%
%
\section{Introduction}

Atmospheric radiation is an important component of global climate.
The ability to study the effects of changes in distribution and balance of radiation are to developing accurate predictive models and forecasts.
Global climate models rely on the efficient computation of radiative transfer at hundreds of points around the globe and many atmospheric levels at each point.
Passive instruments such as AMSU and CrIS require a radiative transfer model to invert radiation measurements and derive atmospheric environmental data such as temperature profiles and water vapor content.
\par In this paper, a simple radiative transfer model is constructed for the purpose of studying the effects of environmental permutations on AMSU observation.
For input, this RTM receives a  single radiosonde file containing pressure, temperature, and dew point information.
By superposition of the radiative absorption from dry air and water vapor, an optical depth can be computed at each layer of atmosphere.
These optical depths are integrated and transformed to total monochromatic transmission from either the top-of-atmosphere or surface.



%%%%%%%%%%%%%%%%%%%%%%%%%%%%%%%%%%%%%%%%%%%%%%%%%%%%%%%%%%%%%%%%%%%%%
% REFERENCES
%%%%%%%%%%%%%%%%%%%%%%%%%%%%%%%%%%%%%%%%%%%%%%%%%%%%%%%%%%%%%%%%%%%%%
 This shows how to enter the commands for making a bibliography using
 BibTeX. It uses references.bib and the ametsoc2014.bst file for the style.

 \bibliographystyle{ametsoc2014}
 \bibliography{references}


%%%%%%%%%%%%%%%%%%%%%%%%%%%%%%%%%%%%%%%%%%%%%%%%%%%%%%%%%%%%%%%%%%%%%
% TABLES
%%%%%%%%%%%%%%%%%%%%%%%%%%%%%%%%%%%%%%%%%%%%%%%%%%%%%%%%%%%%%%%%%%%%%
\begin{table}[h]
\caption{This is a sample table caption and table layout.}\label{t1}
\begin{center}
\begin{tabular}{ccccrrcrc}
\topline
$N$ & $X$ & $Y$ & $Z$\\
\midline
 0000 & 0000 & 0010 & 0000 \\
 0005 & 0004 & 0012 & 0000 \\
 0010 & 0009 & 0020 & 0000 \\
 0015 & 0016 & 0036 & 0002 \\
 0020 & 0030 & 0066 & 0007 \\
 0025 & 0054 & 0115 & 0024 \\
\botline
\end{tabular}
\end{center}
\end{table}
%

\begin{table}
\appendcaption{A1}{Here is the appendix table caption.}
\centering
\begin{tabular}{ccc}
\topline
$1$ & $2$ & $3$ \\
\midline
a&b&c \\
d&e&f \\
\botline
\end{tabular}
\end{table}

%%%%%%%%%%%%%%%%%%%%%%%%%%%%%%%%%%%%%%%%%%%%%%%%%%%%%%%%%%%%%%%%%%%%%
% FIGURES
%%%%%%%%%%%%%%%%%%%%%%%%%%%%%%%%%%%%%%%%%%%%%%%%%%%%%%%%%%%%%%%%%%%%%


\begin{figure}
\centerline{(illustration here)}
\appendcaption{A1}{Here is the appendix figure caption.}
\end{figure}

\begin{figure}
\centerline{(illustration here)}
\appendcaption{B1}{Here is the appendix figure caption.}
\end{figure}



\end{document}
%%%%%%%%%%%%%%%%%%%%%%%%%%%%%%%%%%%%%%%%%%%%%%%%%%%%%%%%%%%%%%%%%%%%%
% END OF AMSPAPER.TEX
%%%%%%%%%%%%%%%%%%%%%%%%%%%%%%%%%%%%%%%%%%%%%%%%%%%%%%%%%%%%%%%%%%%%%