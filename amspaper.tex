%%%%%%%%%%%%%%%%%%%%%%%%%%%%%%%%%%%%%%%%%%%%%%%%%%%%%%%%%%%%%%%%%%%%%%
% amspaper.tex --  LaTeX-based template for submissions to American 
% Meteorological Society journals
%
% Template developed by Amy Hendrickson, 2013, TeXnology Inc., 
% amyh@texnology.com, http://www.texnology.com
% following earlier work by Brian Papa, American Meteorological Society
%
% Email questions to latex@ametsoc.org.
%
%%%%%%%%%%%%%%%%%%%%%%%%%%%%%%%%%%%%%%%%%%%%%%%%%%%%%%%%%%%%%%%%%%%%%
% PREAMBLE
%%%%%%%%%%%%%%%%%%%%%%%%%%%%%%%%%%%%%%%%%%%%%%%%%%%%%%%%%%%%%%%%%%%%%

%% Start with one of the following:
% DOUBLE-SPACED VERSION FOR SUBMISSION TO THE AMS
% \documentclass{ametsoc}

% TWO-COLUMN JOURNAL PAGE LAYOUT---FOR AUTHOR USE ONLY
\documentclass[twocol]{ametsoc}

%%%%%%%%%%%%%%%%%%%%%%%%%%%%%%%%
%%% To be entered only if twocol option is used

\journal{jamc}
\usepackage{gensymb}
\usepackage{graphicx}
\usepackage{booktabs}
\usepackage{hyperref}

\DeclareGraphicsExtensions{.pdf,.png,.jpg}
%  Please choose a journal abbreviation to use above from the following list:
% 
%   jamc     (Journal of Applied Meteorology and Climatology)
%   jtech     (Journal of Atmospheric and Oceanic Technology)
%   jhm      (Journal of Hydrometeorology)
%   jpo     (Journal of Physical Oceanography)
%   jas      (Journal of Atmospheric Sciences)	
%   jcli      (Journal of Climate)
%   mwr      (Monthly Weather Review)
%   wcas      (Weather, Climate, and Society)
%   waf       (Weather and Forecasting)
%   bams (Bulletin of the American Meteorological Society)
%   ei    (Earth Interactions)

%%%%%%%%%%%%%%%%%%%%%%%%%%%%%%%%
%Citations should be of the form ``author year''  not ``author, year''
\bibpunct{(}{)}{;}{a}{}{,}

%%%%%%%%%%%%%%%%%%%%%%%%%%%%%%%%

%%% To be entered by author:

%% May use \\ to break lines in title:

\title{Weighting Function Project}

%%% Enter authors' names, as you see in this example:
%%% Use \correspondingauthor{} and \thanks{Current Affiliation:...}
%%% immediately following the appropriate author.
%%%
%%% Note that the \correspondingauthor{} command is NECESSARY.
%%% The \thanks{} commands are OPTIONAL.

    \authors{Coda Phillips\correspondingauthor{Coda Phillips, 
     University of Wisconsin - Madison 
     }}

     \affiliation{University of Wisconsin - Madison}

\email{codaphillips@gmail.com}



%%%%%%%%%%%%%%%%%%%%%%%%%%%%%%%%%%%%%%%%%%%%%%%%%%%%%%%%%%%%%%%%%%%%%
% ABSTRACT
%
% Enter your Abstract here

\abstract{I make a radiative transfer model for AMSU and study things} 

\begin{document}


%% Necessary!
\maketitle


%%%%%%%%%%%%%%%%%%%%%%%%%%%%%%%%%%%%%%%%%%%%%%%%%%%%%%%%%%%%%%%%%%%%%
% MAIN BODY OF PAPER
%%%%%%%%%%%%%%%%%%%%%%%%%%%%%%%%%%%%%%%%%%%%%%%%%%%%%%%%%%%%%%%%%%%%%
%
\section{Introduction}

Atmospheric radiation is an important component of global climate.
The ability to study the effects of changes in distribution and balance of radiation are to developing accurate predictive models and forecasts.
Global climate models rely on the efficient computation of radiative transfer at hundreds of points around the globe and many atmospheric levels at each point.
Passive instruments such as AMSU and CrIS require a radiative transfer model to invert radiation measurements and derive atmospheric environmental data such as temperature profiles and water vapor content.
\par In this paper, a simple radiative transfer model is constructed for the purpose of studying the effects of environmental permutations on AMSU observation.
For input, this RTM receives a  single radiosonde file containing pressure, temperature, and dew point information.
By superposition of the radiative absorption from dry air and water vapor, an optical depth can be computed at each layer of atmosphere.
These optical depths are integrated and transformed to total monochromatic transmission from either the top-of-atmosphere or the surface.

\section{Methods}

For all experiments, surface temperature was assumed equal to the temperature of the lowest radiosonde measurement.

The surface, in all cases except the emissivity experiments, is assumed to radiate as a blackbody at thermal equilibrium. That is, the emissivity is equal to 1 at all wavelengths.

The frequencies corresponding to the referenced AMSU channels are defined in \autoref{tab:amsua} and \autoref{tab:amsub}.

\begin{table}
	\centering
	\caption{AMSU-A Channels}
	\begin{tabular}{cc}
		\toprule
		Channel	& Frequency (GHz)\\
		\midrule
	    3 & 50.3000 \\
	    4 & 52.8000 \\
	    5 & 53.7110 \\
	    6 & 54.4000 \\
	    7 & 54.9400 \\
	    8 & 55.5000 \\
	    9 & 57.2900 \\
	    10 & 57.5070 \\
	    \bottomrule
	\end{tabular}
	\label{tab:amsua}
\end{table}

\begin{table}
	\centering
	\caption{AMSU-B Channels}
	\begin{tabular}{cc}
		\toprule
		Channel	& Frequency (GHz)\\
		\midrule
		16 & 89.00 \\
		17 & 150.0 \\
		18 & 184.31 \\
		19 & 186.31 \\
		20 & 190.31 \\
	    \bottomrule
	\end{tabular}
	\label{tab:amsub}
\end{table}

\subsection{Atmospheric Profile}

Before an atmospheric profile can be computed, environmental data must be ingested from a radiosonde file.
Contained in the radiosonde data is a temperature and dew point profile at various pressures.

\subsection{Layers}

A model atmosphere is partitioned into homogeneous atmospheric layers, each possessing the mean value of environmental variables and a geometric height.
The geometric height associated with each layer is a required to calculate optical depths.
This is derived from a form of the hydrostatic equation rather than being obtained from the radiosonde's GPS altitude.

\subsection{Absorption}

The total optical depth, $\tau$, of each layer is computed by adding optical depth components from dry air and water vapor.
These components are computed from the product of gas partial mass and mass extinction coefficient.

\subsection{Transmission}

An integral over optical depths is performed by computing a cumulative sum.
According to Beer's law, the antilogarithm of these depths results in the total transmission to that layer from either the top-of-atmosphere or surface, depending on the direction of the cumulative sum.
By Schwarzschild's equation, the weighting function takes the value of the derivative of transmission at each level.
A discrete difference approximates this derivative.

\subsection{Weights}

The resulting weights associated with each layer can finally be divided by geometric depth their respective layers to derive the weight as a function of geometric depth.

\subsection{Experiments}

\paragraph*{Case Study} For the following experiments, two cases of atmosphere will be compared.
One case will study the Arctic atmosphere and the other will study the Tropical.
The Arctic case uses a radiosonde launched from Churchill, Canada at 12Z on the first of March, 2012.
The Tropical radiosonde was launched from Boa Vista, Brazil, at the same date and time.

\paragraph{Low frequency channels (3-10) Module A }
 
\paragraph{High frequency channels (18-20)}


\paragraph{Spectra}
Compute and plot (on the same plot) the complete spectra of microwave brightness temperature for the two soundings — 0.1 to 300 GHz at intervals of 0.1 GHz.   Describe the differences in results. 
Attribute differences to (a) differences in the surface temperature, (b) differences in the atmospheric temperature profile, and (c) differences in the atmospheric humidity profile.

The lack of continuum absorption in the Canadian atmosphere is supported by the fact that little water vapor is present in that case.

\paragraph{Emissivity}
For just the tropical sounding, plot the spectra for two cases:  emissivity equal to 1  and emissivity equal to 0.5.  What do your results tell you about the ranges of frequencies for which the atmosphere is opaque?
Repeat the above for the polar sounding and describe what changes relative to the previous results.

\paragraph{Elevation angle}
Again, plot spectra for the tropical sounding, this time for two nadir angles θ:  0 degrees and 55 degrees.  Discuss the resulting differences.

\section{Results}

\subsection{AMSU-A Weighting Functions}

Shown in \autoref{fig:weighting_a} are the weighting functions for the AMSU-A channels for both the Canadian and Brazilian cases.
These results show some similarity between the cases in the relative magnitudes and distributions of weights.
However, the peak weights in Canadian case almost always occur at a lower altitude than those of the Brazilian case.
These depressed heights indicate a comparatively more transparent Canadian atmosphere.
Some channels are much more opaque than others, despite all inhabiting a relatively narrow bandwidth.

\begin{figure}
	\centering
	\includegraphics[width=.5\textwidth]{figures/weighting_a}
	\caption{thing1}
	\label{fig:weighting_a}
\end{figure}

\subsection{AMSU-B Weighting Functions}

The weights shown in \autoref{fig:weighting_b} demonstrate that little absorption occurs in the upper atmosphere at these higher frequencies.
The difference between the weight-peak altitude in the Brazilian and Canadian atmospheres is much more pronounced here than with the AMSU-A channels.
These data would suggest that although a high amount of absorption occurs, it is relegated to the troposphere.
It is likely that almost all absorption in these bands is driven by water vapor, which explains why radiation is able to propagate farther in the dry Canadian atmosphere before it is extinguished.

\begin{figure}
	\centering
	\includegraphics[width=.5\textwidth]{figures/weighting_b}
	\caption{thing2}
	\label{fig:weighting_b}
\end{figure}

\subsection{Spectra}

The predicted spectra for Brazil and Canada is shown in \autoref{fig:spec}.
Excluding the strongest absorption lines, the predicted brightness temperatures observed from space for the Brazilian and Canadian cases differ significantly.
The average values of brightness temperature for the Brazilian case are nearly 40~K greater than those from Canada.
Additionally, there is a downward slope of brightness temperatures from Brazil as frequency increases.
This effect is not apparent in the Canadian case.
Furthermore, the absorption band of oxygen at 22~GHz does not produce a noticeable effect in the Canadian brightness temperatures, but it appears as a small dip in those from Brazil.
Within the strongest absorption lines, these spectra look remarkably similar.

\begin{figure}
	\centering
	\includegraphics[width=.5\textwidth]{figures/spec}
	\caption{spectrum}
	\label{fig:spec}
\end{figure}

\subsubsection{Surface Temperature}

The Brazilian surface temperature is 296.75~\degree{K}, wherease the Canadian surface temperature is 254.05~\degree{K}.
The surface temperature in Canada is much colder than in Brazil.
This effect is evident in the satellite spectrum at the lowest frequencies, where the atmosphere should be transparent.
The resulting brightness temperatures are nearly identical to the actual temperatures.
This correlation would suggest that both atmospheres are highly transparent in these bands, and changes in brightness temperature emitted from the surface produce a strong effect in predicted brightness temperatures.
However, surface temperature cannot explain the difference in brightness temperature response to frequency or the values within absorption bands.

\subsubsection{Temperature Profile}

It appears that Canadian temperatures are warmer than Brazilian temperatures at 17.5~km.
But, as expected, temperatures are colder in Canada below tropopause.
In the weighting functions you can see the altitude contributing the most to absorption and emission. Therefore, brightness temperatures will tend toward the physical temperature of that particular layer rather than the others.
In channel~18, corresponding to 184~GHz, the altitude of peak absorption is located at about 10~km in Brazil and 3~km in Canada.
If brightness temperature is read at 184~GHz, it comes out to about 240~K for both Brazil and Canada.
This matches the temperatures in the profile measured at the altitudes of peak emission, which confirms that the radiative transfer model is consistent.

You can see the Brazilian brightness temperatures diminish as continuum absorption increases with higher frequencies.
Outside major oxygen and water vapor absorption bands, absorption is mostly governed by the water vapor continuum absorption.


\begin{figure}
	\centering
	\includegraphics[width=.5\textwidth]{figures/tdry}
	\caption{Temperature}
	\label{fig:tdry}
\end{figure}

\subsubsection{Humidity}

Brazil has much more moisture than Canada

Water vapor concentrated in the troposphere and peaking at a few km altitude.

Brazil total precipitable water much greater

brazil: 52.22 kg/$m^2$

canada: 3.12 kg/$m^2$

Water vapor is a strong absorber, it has one particularly strong band at 183 GHz

Oxygen is responsibile for the bands seen at 20, 60, and 118 GHz

\begin{figure}
	\centering
	\includegraphics[width=.5\textwidth]{figures/qbar}
	\caption{Specific Humidity}
	\label{fig:qbar}
\end{figure}

\subsection{Emissivity}


\begin{figure}Most of the difference occurs at lower frequencies
The received brightness temperature is much higher when viewed at a slant. With the greater optical distance, a much larger component of brightness temperature seems to be coming from the surface. Reflected brightness temperatures off the surface are presumed to be optically thicker as well. This all translates to a greater amount of flux being available at the surface which can be detected by the satellite throught the relatively transparent atmosphere.
Oxygen absorption at 60 GHz remains about constant. Most radiation in this band is absorbed at high altitudes (in excess of 20km) as can be seen in the channel 10 weighting function.
	\centering
	\includegraphics[width=.5\textwidth]{figures/brazil_emiss}
	\caption{Brazil emissivity}
	\label{fig:bemiss}
\end{figure}


\begin{figure}
	\centering
	\includegraphics[width=.5\textwidth]{figures/canada_emiss}
	\caption{Canada emissivity}
	\label{fig:cemiss}
\end{figure}

\subsubsection{$\epsilon = 1$}
For just the tropical sounding, plot the spectra for two cases:  emissivity equal to 1  and emissivity equal to 0.5.
\subsubsection{$\epsilon = 0.5$}
What do your results tell you about the ranges of frequencies for which the atmosphere is opaque?
Repeat the above for the polar sounding and describe what changes relative to the previous results.

\subsection{Elevation}


\begin{figure}
	\centering
	\includegraphics[width=.5\textwidth]{figures/brazil_elevation}
	\caption{Elevation angle}
	\label{fig:elevation}
\end{figure}

Again, plot spectra for the tropical sounding, this time for two nadir angles θ:  0 degrees and 55 degrees.  Discuss the resulting differences.
\subsubsection{$\theta = 0 \degree{} $}
\subsubsection{$\theta = 55 \degree{} $}

Most of the difference occurs at lower frequencies

The received brightness temperature is much higher when viewed at a slant. With the greater optical distance, a much larger component of brightness temperature seems to be coming from the surface. 

Reflected brightness temperatures off the surface are presumed to be optically thicker as well. This all translates to a greater amount of flux being available at the surface which can be detected by the satellite throught the relatively transparent atmosphere.

Oxygen absorption at 60 GHz remains about constant. Most radiation in this band is absorbed at high altitudes (in excess of 20km) as can be seen in the channel 10 weighting function.


\section{Conclusion}

%%%%%%%%%%%%%%%%%%%%%%%%%%%%%%%%%%%%%%%%%%%%%%%%%%%%%%%%%%%%%%%%%%%%%
% REFERENCES
%%%%%%%%%%%%%%%%%%%%%%%%%%%%%%%%%%%%%%%%%%%%%%%%%%%%%%%%%%%%%%%%%%%%%
 This shows how to enter the commands for making a bibliography using
 BibTeX. It uses references.bib and the ametsoc2014.bst file for the style.

 \bibliographystyle{ametsoc2014}
 \bibliography{references}

\end{document}
%%%%%%%%%%%%%%%%%%%%%%%%%%%%%%%%%%%%%%%%%%%%%%%%%%%%%%%%%%%%%%%%%%%%%
% END OF AMSPAPER.TEX
%%%%%%%%%%%%%%%%%%%%%%%%%%%%%%%%%%%%%%%%%%%%%%%%%%%%%%%%%%%%%%%%%%%%%